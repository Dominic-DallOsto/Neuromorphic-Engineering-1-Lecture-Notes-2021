\section*{Author Contribution}

The project of writing these lecture notes emerged from discussions about the challenging NE1 course between Lioba Schürmann and Yassine Taoudi Benchekroun in early December 2021. Though our original ambition was to better prepare for the exam, it quickly became a project with bigger purpose: building a digestible guide to all present and future students of the challenging NE1 course, and Neuromorphic Engineering as a field. 
This shift of purpose is a result of the passion and ambition that we feel towards this fascinating field of research - and it was particularly emphasized when Wencan Huang and Pietro Bonazzi joined us in early January, as they shared our enthusiasm for the project. Pietro and Wencan joining us truly gave another breath to this project. 

We now all envision this as an \textit{open source project} and very much look forward to adding contribution from students. We invite all readers to point us to the details we may have omitted, to the elements we may have described in inadequate fashion, or anything that could improve the overall quality of this humble work. It will be our honour to receive your contribution and suggestions and add you to our list of contributors. 

The individual contribution of authors is as follows: 

\begin{itemize}
    \item Organization, coordination and structuring: 
    \begin{itemize}
        \item Yassine Taoudi Benchekroun
    \end{itemize}
    \item Chapter Authorship: 
    \begin{itemize}
        \item Lioba Schurmann: Chapter 3 (Transistor Operation), Chapter 8 (Silicon Synapse) and Chapter 9 (Silicon Neuron).
        \item Yassine Taoudi Benchekroun: All other chapters.
    \end{itemize}
    \item Lab takeaways: 
    \begin{itemize}
        \item Wencan Huang: Odd-numbered labs.
        \item Pietro Bonazzi: Even-numbered labs.
    \end{itemize} 
    \item Editing:
    \begin{itemize}
        \item All others contributed to editing and reviewing each other's part. Wencan Huang was particularly helpful in helping with the device physics and transistor operation sections, which were the most challenging to write of the whole document.
    \end{itemize}
\end{itemize}
