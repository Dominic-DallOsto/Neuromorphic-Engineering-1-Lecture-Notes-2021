\subsection{Bulks, wells and biasing the MOSFET Bulks}
\newline \newline
\textbf{What's up with bulks and wells:} \\
Good question. A lot is actually going on, but let's try to keep it short and simple. Remember from chapter 0 (which I hope you reviewed!) that voltage is all about potential \emph{difference}, so you need "point of references" that are somewhat common to everywhere in your circuit. It so happens that in a circuit, all transistors do not necessarily share the same bulk, which is typically the natural reference potential of MOSFETS. We therefore need a common reference potential that multiple transistors can have access to: this is called the common bulk potential. There are two common bulk potentials in a circuit: the lowest (ground) is denoted $V_{ss}$ and the highest is denoted $V_{dd}$.
You never want charge carriers from the source/drain of transistor to go into the bulks (except in very rare scenarios not considered in this course) - to avoid that, source and drain diodes are \textbf{reverse-biased} to the bulks. To do so, in an nFET, the bulk is typically at $V_{ss}$ and source will always be slightly higher (or equal) potential: $V_s \geq V_ss$. This ensures the reverse bias as we apply a higher potential to the n-doped semi conductor than the p-doped semi conductor it is in contact with. Same holds in PFET, but in reverse.
All voltages in the circuit are references to $V_{ss}$, which is the 0 V and lowest potential of our circuit - so all voltages we'll be dealing with are typically positive. Connections to $V_{ss}$ are marked as connection to ground, and connections to $V_{dd}$ with a slanting line. 
