\subsection{Test Yourself}

You should be able to answer the following questions for the exam (mainly taken from the winter study sheet).

\begin{itemize}
    \item What does it mean for a MOS transistor channel to be accumulated, flat-band, depleted, inverted?
    \item Knowledge of how subthreshold transistor operation is a diffusion process and why it
depends exponentially on the terminal voltages.
    \item What is the meaning of ”saturation”?
    \item What is the triode or linear operating range?
    \item $I_{ds}$ vs $V_{gs}$ on log scale.
    \item Differences between n- and p-fets.
    \item Typical values of $I_0$, $\kappa$ and subthreshold operating range.
    \item What are wells and how should the wells be biased relative to the substrate?
    \item What is the ”back gate” or ”body effect”?
    \item How is the back gate is related to $\kappa$?
    \item How to make a MOS capacitor and what is its C-V relationship.
    \item How transistors work above threshold.
    \item What is the linear or triode region and what is the saturation region?
    \item How do they depend on gate and threshold voltage?
    \item How the Early effect comes about.
    \item Typical values for Early voltage.
    \item How to sketch graphs of transistor current vs. gate voltage and drain-source voltage.
    \item How above-threshold transistors go into saturation and why the saturation voltage is equal to the gate overdrive.
    \item The above-threshold current equations.
    \item How above-threshold current depends on Cox and mobility.
    \item What is DIBL (drain induced barrier lowering) and II (impact ionization)?
    \item How transconductance and drain resistance combine to generate voltage gain and what is the intrinsic voltage gain of a transistor.
    \item How transconductance and drain resistance combine to generate voltage gain.
\end{itemize}