\subsection{Prelude: Motivation}

\begin{quote}
    \textit{"Over the past 600 million years, biology has solved the problem of processing massive amounts of noisy and highly redundant information in a constantly changing environment by evolving networks of billions of highly interconnected nerve cells. It is the task of scientists—be they mathematicians, physicists, biologists, psychologists, or computer scientists—to understand the principles underlying information processing in these complex structures. At the same time, researchers in machine vision, pattern recognition, speech understanding, robotics, and other areas of artificial intelligence can profit from understanding features of existing nervous systems. Thus, a new field is emerging: the study of how computations can be carried out in extensive networks of heavily interconnected processing elements, whether networks are carbon - or silicon-based."}\footnote{From Carver Mead's Textbook - Foreword. 1989.}
\end{quote}

One of the key motivations behind the idea of Neuromorphic Engineering is to emulate the most efficient and interesting brain processes to electronics. It is highly desirable to copy the brain as it has gone through thousands of years of natural selection, it is thus likely that it would do a lot of things better than what we'd manage to engineer while overlooking it. Vision is one of the most obvious and most accessible process to emulate: the retina has been extensively studied and presents a rich myriad of layered elements that were understood to be reproducible with electronics. As mentionned in chapter 0, Misha Mahowald - in Carver Mead's lab - first brought the idea and successfully developped it. This section aims at understanding the subelements that constitute vision. We'll first look at biological vision through a brief overview of the brain and the retina. We will then proceed to study how light works and can be processed with our favourite material: silicon. Finally, we'll look at some very basic circuits that constitute building blocks of neuromorphic vision.   

