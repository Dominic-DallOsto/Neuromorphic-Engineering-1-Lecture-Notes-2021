\subsection{Laboratory : Transistors above threshold}

As you have understood from reading the theory, current flows through drift and diffusion.

In this section, we will study p-type and n-type transistors when drift is the dominant driver of current. 

\subsubsection{Voltage threshold and Beta}

To find a current dominated by drift, we should look to a gate voltage that is higher than a certain threshold. That's why drift or strong inversion is also described as above threshold operation. It needs to be above (or below in p-fet) a certain threshold and from there we can extract the current driven by drift. This should ring a bell \ref{chapter:transistor}. 

Source and drain are initialized differently depending whether we are interested in the ohmic or in the saturation region. 
To study saturation region, the full range of voltage operations is taken, whereas, in the ohmic region, source and drain voltages should different only slightly (i.e. 0.2 V). This should also ring a bell \ref{chapter:transistor}. 

We study a range of different gates voltages. Refers to the plot below, to see a real  n-FET in ohmic and saturation region. 

If you have something similar to this, well done ! You can basically control electricity with your fingers. 

The finding are very fundamental here. We proved that with a certain source and drain voltages, and a varying gate voltage, current flows in a transistor following a certain curve.
We realized that the equation are only approximation of reality, but they are good approximation of reality.

It has been found that for an n-fet, the voltage gate threshold (the point where the current suddenly increases), was around ~0.8V. 

Saturation regions start at this index and ohmic regions end at this threshold. 

To estimate the real beta from measurement, first, you need to compute the slope of the curve and then apply some transformation to it (the transformation are derived directly from the formulas in \ref{section:aboveTHR}). 

Betas in ohmic region has been found to be a float between 2 and 3, whereas in saturation region they were near 0 . 

The betas of a p-fet and an n-fet, the ratio between them should ideally be one. If it's not one, but it's approximately one (~0.9), then don't worry, but think : what could be the reason for this discrepancy?

\subsubsection{Early voltage}

To measure the Early voltage, in n-fet, you need to vary the voltage gate (similarly as before, within a range) and vary the voltage at the drain. This way we can get more data and test whether in reality the Early Voltage is the same with different drain voltages. 

You should get something like [Figure].

By fitting a line in the saturation region, you can extract the early voltage. Remember the definition of Early Voltage in \ref{section:aboveTHR} . 

You will find there is no single Early Voltage, but the measurements you get should be within one order of magnitude.
Attention, if you measured a voltage gate of 1.8V than the saturation region will be forced to stay flat by the built in system control, avoid using 1.8V for the gate. 
When the voltage gate goes up the saturation region is less flat and the absolute early voltage gets bigger.

